This section describes the sound program, and details how it was developed. This section covers procedure, setup and configuration, tools and program details.

need to include a bit about the Makefile here


\section{Functional Description}
	The solution program plays sound effects and music, and is controlled by the eight buttons on the STK1000.
The LEDs are used to indicate which sound is playing.

When the program is started, the board is in idle mode, ready to react to button presses.
Pressing any of the buttons \texttt{SW0}-\texttt{SW3} plays a piece of music, which loops until another sound is selected.
Pressing any of the buttons \texttt{SW4}-\texttt{SW6} plays a sound effect, which is not looped.
Pressing \texttt{SW7} stops all playback.

\subsection{Sound effects}

The sound effects are generatively composed by wrapping a generator signal in a configurable ADSR volume envelope.
Because the internal DAC of the STK1000 is rather noisy because of signal bleeding and poorly implemented hardware filters, the sound from the generators can sound rather dirty.

The available generator signals in the program are NOISE, SAWTOOTH and SQUARE.

A sawtooth wave is a wave that increases linearly, until it cuts off, as seen in figure \ref{img-sw5zoom}.
The sawtooth wave includes all the integer harmonics for the given frequencies.
\begin{figure}[H]
	\includegraphics[width = \textwidth]{images/SW5zoom.png}
	\caption{A noisy sawtooth wave recorded from the STK1000.}
	\label{img-sw5zoom}
\end{figure}

An ideal square wave is either at a maximum or minimum amplitude as seen in figure \ref{img-sw4zoom}, and shifts between them instantly.
Unlike a sawtooth wave, the square wave only contains odd-numbered integer harmonics.
\begin{figure}[H]
	\includegraphics[width = \textwidth]{images/SW4zoom.png}
	\caption{A noisy square wave recorded from the STK1000.}
	\label{img-sw4zoom}
\end{figure}

A noise is rather the lack of a waveform, with randomly chosen samples, as depicted in figure \ref{img-sw6zoom}.
Noise can be described as the sound your tv makes when you tune into frequencies that there is no broadcasting on.
\begin{figure}[H]
	\includegraphics[width = \textwidth]{images/SW6zoom.png}
	\caption{A noise signal recorded from the STK1000}
	\label{img-sw6zoom}
\end{figure}

In order to make sound effects we need more than just pure waves, as sound almost never consists of just a wave with constant amplitude.
It is the job of the ADSR envelope to solve this issue.
An envelope sets bounds for the amplitude of a wave, and an ADSR envelope divides the wave into four parts: attack, decay, sustain and release.

During the attack, the amplitude is gradually increased from 0 to a maximum value.
After the attack, the amplitude gradually decays to a sustain level, which is typically a fraction of the maximum value.
The sound stays at the sustain level for a given time, until it is released, and the amplitude gradually decreases until it reaches 0 again.

\subsubsection{Explosion}

`Explosion' is a NOISE-based sound effect with the following ADSR envelope:
\begin{itemize}
	\item{Attack: 0 ms}
	\item{Decay: 1000 ms}
	\item{Sustain: 0\%}
	\item{Release: 0 ms}
\end{itemize}
The effect is held for 0 ms.
The total length of the effect is $0 ms + 1000 ms + 0 ms + 0 ms = 1000 ms$.
`Explosion' can be triggered by pressing \texttt{SW6}.
The sound effect is depicted in figure \ref{img-sw6}.

\begin{figure}[H]
	\includegraphics[width = \textwidth]{images/SW6.png}
	\caption{An overview of the Explosion sound effect, recorded from the STK1000.}
	\label{img-sw6}
\end{figure}


\subsubsection{Air horn}
`Air horn' is a SAWTOOTH-based sound effect with the following ADSR envelope:
\begin{itemize}
	\item{Attack: 100 ms}
	\item{Decay: 100 ms}
	\item{Sustain: 70\%}
	\item{Release: 500 ms}
\end{itemize}
The effect is held for 0 ms.
The total length of the effect is $100 ms + 100 ms + 500 ms + 0 ms = 700 ms$.
`Air horn' can be triggered by pressing \texttt{SW5}.
See figure \ref{img-sw5} for a visualization of `Air horn'.

\begin{figure}[H]
	\includegraphics[width = \textwidth]{images/SW5.png}
	\caption{An overview of the Air horn sound effect, recorded from the STK1000.}
	\label{img-sw5}
\end{figure}

\subsubsection{Teleport}
`Teleport' is a SQUARE-based sound effect with the following ADSR envelope:
\begin{itemize}
	\item{Attack: 500 ms}
	\item{Decay: 1250 ms}
	\item{Sustain: 20\%}
	\item{Release: 250 ms}
\end{itemize}
The effect is held for 0 ms.
The total length of the effect is $500 ms + 1250 ms + 250 ms + 0 ms = 2000 ms$.
`Teleport' can be triggered by pressing \texttt{SW4}.
A picture of the recorded sound can be seen in figure \ref{img-sw4}.

\begin{figure}[H]
	\includegraphics[width = \textwidth]{images/SW4.png}
	\caption{An overview of the Teleport sound effect, recorded from the STK1000.}
	\label{img-sw4}
\end{figure}


\subsection{Music}

The music pieces in the solution program are played by the MOD player.

\subsubsection{Tuulenvire by Dizzy/CNCD}
\emph{Tuulenvire} is a 2:09 long 808KB composition in the ambient genre, featuring piano and accordion, amongst other instruments.
This composition was chosen to demonstrate how careful composing can render realistic compositions with a relatively small memory footprint.
It uses 25 different PCM-coded sounds.
Tuulenvire can be triggered by pressing \texttt{SW3}.

\subsubsection{Boesendorfer P. S. S. by Romeo Knight}
\emph{Boesendorfer P. S. S.} is a 3:22 long 211KB solo piano composition, chosen to illustrate the possibilities enabled by a hybrid generative/recorded approach.
It uses 9 different PCM-coded sounds.
Boesendorfer P. S. S. can be triggered by pressing \texttt{SW2}.

\subsubsection{Drop The Panic by H0ffmann}
\emph{Drop The Panic} is a 4:05 long 702KB ``glitch-hop'' composition.
It was chosen to show how MOD files can support embedded vocals.
It uses 31 different PCM-coded sounds.
The composition was tweaked by adding some extra inaudible notes in the beginning of the song to decrease critical cache misses by the MOD player during playback on the STK1000.
Drop The Panic can be triggered by pressing \texttt{SW1}.

\subsubsection{Bacongrytor by Maktone}
\emph{Bacongrytor} is a 15Kb endless loop chiptune-style composition, chosen to demonstrate the compactness of the MOD format, and therefore its aptfulness for use on microcontrollers.
It uses 7 different PCM-coded sounds.
Bacongrytor can be triggered by pressing \texttt{SW0}.




\section{Solution Components}
	This section describes the solutions' various components.

\subsection{Linux Device Drivers for the STK1000}
	Something about LDDs?
	Init & Exit functions and cleanup and such

	\subsubsection{LED driver}
		the LED driver allows the software to turn the LEDs on and off?
	\subsubsection{Button driver}
		the buttan driver reads the buttons' state.	

	\subsubsection{Sound driver}
		/dev/dsa!
	\subsubsection{Display driver}
		/dev/fb0
		Screen size: 320x240
		32 bits per pixel (8bit color depth)

\subsection{Game Engine -- WORKING TITLE}
	this is the game engine we developed. it has parts that are pretty good. like sound, it can play sound and display things on the screen.
	\subsubection{Sound}
		how does the engine handle sound?
		we just pipe that to /dev/dsa right?
		pretty much I guess.
	\subsubsection{Graphics}
		them graphics are pretty good but how do we handle them????

\subsection{The .sm file format}
	yo imma just rewrite http://www.stepmania.com/wiki/The_.SM_file_format

\section{Configuration}
	We can reference our previous report here, so that we don't have to write so much.

\subsection{Jumpers}

The jumpers were set up as in the previous assignment.\cite{tdt4258-1}

\subsection{GPIO connections}

The GPIO connections were set up as in the previous assignment.\cite{tdt4258-1}

\subsection{Audio}

Headphones were connected to the board's audio jack connector as in figure \ref{img-audiojack}, so people can listen to the sweet, sweet sounds of the solution program.

\begin{figure}[h]
\includegraphics[width = \textwidth]{images/audiojack.png}
\caption{The audio jack port on the STK1000 before and after a headphone is connected.}
\label{img-audiojack}
\end{figure}

\newpage

\section{Development of the program}
% Experimental Procedure


\subsection{Sound effect synth}

\subsection{Libmodam}

Libmodam was initially developed in a linux environment on an x86 PC, and later ported to avr32 and the STK1000.
This was done for comfort reasons: there was no need to be in the lab to develop; testing and iterating went a lot faster when code and data didn't need to be uploaded after each modification; and there was no need to focus on performance until features were confirmed to be working properly.
As an added bonus, this also ensured a certain degree of portability, which is a nice property for a library to have.
The python file conversion tool was developed alongside libmodam.

When the code was run on the STK1000 for the first time, the performance difference between the Intel i7 multi-GHz linux laptop and the STK1000 became immediately apparent.
The sound produced by libmodam on the STK1000 at this point did not even remotely sound like music.
In fact, it was more akin to a bowl of half-eaten oatmeal porridge left overnight, were porridge as audibly dull as it is bland in taste.
One of the main reasons for the abysmal performance on the STK1000 was the lack of a floating point unit in the AP7000.
Libmodam used floating point numbers heavily.
The library was rewritten to use integer arithmetic, which helped immensely.
Further optimizations were iteratively applied after this, until the code reached a point of acceptable performance.

\subsection{The main program}

How everything was assembled in the main program.

\subsubsection{Setting up the LEDs}
\subsubsection{Setting up the buttons}
\subsubsection{Setting up the audio}
rant about sample rates, div, diven


list of what we did:

* Board was set up w/ jumpers and such
* Leds and buttons were hooked up in hardware
* Leds and buttons were hooked up in software
* Code was split into separate files
* Audio was hooked up with apropriate settings
* Sound was tested to work using random noise


the following two groups of bullet points happened in parallel:

* a C sound effect synth inspired by sfxr was prototyped on a PC
* the synth was ported to avr32
* the synth was developed further on the avr32
* the synth was used in the stk1000 program to play various sound effects

* a C MOD player library + python tools inspired by amiga trackers were developed on a PC
* the library was tested on the avr32 and needed a great deal of optimization
* a great deal of optimization occurred 
* the library was used in the stk1000 program to play selected mods


finally:

* the actual main program flow was decided upon and written, hooking button and led behaviour together with sound effects and music


\section{Programming Environment}
	\subsection{Allegro}
	\texttt{Allegro 4.2} was used to emulate the STK1000 screen in order to run and test the engine code without direct access to the STK1000.
	For more information about Allegro, see \url{https://www.allegro.cc/}

\subsection{Make}
	See \cite{tdt4258-1} for a brief description of Make.

\subsection{Other tools}
	\begin{itemize}
		\item{\texttt{Vim} was employed as the authors' text editor of choice.}
		\item{\texttt{git} was used for version control during the development of the solution.}
		\item{The project is hosted in a private GitHub repository.}
		\item{The report was written with \LaTeX.}
		\item{\texttt{oggdec} from \texttt{vorbis-tools v1.4.0} was used to convert \texttt{.ogg} files to raw audio files.}
		\begin{itemize}
				\item{\url{http://www.rarewares.org/ogg-tools.php}}
		\end{itemize}
		\item{\texttt{gparted} was used to format the SD Card before installing Linux on it.}
		\item{The \texttt{U-Boot} bootloader was used to boot linux from an SD card.}
		\item{The Atmel AVR32 toolkit was used for compiling, linking etc. the different programs for the STK1000.}
		\item{\texttt{Texture Font Generator (ssc)} -- a texture font generator bundled with Stepmania \texttt{5.0 beta 1a} -- was used to generate font textures.}
		\item{\texttt{Adobe PhotoShop CS6} was used to create static imagery for the game interface.}
		\item{\texttt{GIMP} was used to create static imagery for the game interface.}
	\end{itemize}
