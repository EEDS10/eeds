
This section describes how the sound program was developed. Ut covers procedure, setup and configuration, tools and program details.



need to include a bit about the Makefile here


Description parts

General information about sound
digital vs analog sound

different ways of storing sounds
samples ('bitmaps')
constructive instructions ('vectors')
a combination of the two


Development of the program

list of what we did:

* Board was set up w/ jumpers and such
* Leds and buttons were hooked up in hardware
* Leds and buttons were hooked up in software
* Code was split into separate files
* Audio was hooked up with apropriate settings
* Sound was tested to work using random noise


the following two groups of bullet points happened in parallel:

* a C sound effect synth inspired by sfxr was prototyped on a PC
* the synth was ported to avr32
* the synth was developed further on the avr32
* the synth was used in the stk1000 program to play various sound effects

* a C MOD player library + python tools inspired by amiga trackers were developed on a PC
* the library was tested on the avr32 and needed a great deal of optimization
* a great deal of optimization occurred 
* the library was used in the stk1000 program to play selected mods


finally:

* the actual main program flow was decided upon and written, hooking button and led behaviour together with sound effects and music



Configuration

We can reference our previous report here, so that we don't have to write so much.


Programming environment

JTAGICE
We can reference our previous report here, so that we don't have to write so much.

GNU Debugger
since last time:
* discovered tui mode: looks nice, breaks the makefile

Make
we can reference.

Other tools
* OpenMPT was used to examine MOD files for libmodam
* vim
* git
* github
* latex
* avr32 toolchain

Setting up the LEDs
Setting up the buttons
Setting up the audio


Program flow

* main program
* synth
* libmodam
