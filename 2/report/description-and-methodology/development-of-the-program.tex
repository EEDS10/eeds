
\subsection{Sound effect synth}

\subsection{Libmodam}

Libmodam was initially developed in a linux environment on an x86 PC, and later ported to avr32 and the STK1000.
This was done for comfort reasons: there was no need to be in the lab to develop; testing and iterating went a lot faster when code and data didn't need to be uploaded after each modification; and there was no need to focus on performance until features were confirmed to be working properly.
As an added bonus, this also ensured a certain degree of portability, which is a nice property for a library to have.
The python file conversion tool was developed alongside libmodam.

When the code was run on the STK1000 for the first time, the performance difference between the Intel i7 multi-GHz linux laptop and the STK1000 became immediately apparent.
The sound produced by libmodam on the STK1000 at this point did not even remotely sound like music.
In fact, it was more akin to a bowl of half-eaten oatmeal porridge left overnight, were porridge as audibly dull as it is bland in taste.
One of the main reasons for the abysmal performance on the STK1000 was the lack of a floating point unit in the AP7000.
Libmodam used floating point numbers heavily.
The library was rewritten to use integer arithmetic, which helped immensely.
Further optimizations were iteratively applied after this, until the code reached a point of acceptable performance.

\subsection{The main program}

How everything was assembled in the main program.

\subsubsection{Setting up the LEDs}
\subsubsection{Setting up the buttons}
\subsubsection{Setting up the audio}
rant about sample rates, div, diven


list of what we did:

* Board was set up w/ jumpers and such
* Leds and buttons were hooked up in hardware
* Leds and buttons were hooked up in software
* Code was split into separate files
* Audio was hooked up with apropriate settings
* Sound was tested to work using random noise


the following two groups of bullet points happened in parallel:

* a C sound effect synth inspired by sfxr was prototyped on a PC
* the synth was ported to avr32
* the synth was developed further on the avr32
* the synth was used in the stk1000 program to play various sound effects

* a C MOD player library + python tools inspired by amiga trackers were developed on a PC
* the library was tested on the avr32 and needed a great deal of optimization
* a great deal of optimization occurred 
* the library was used in the stk1000 program to play selected mods


finally:

* the actual main program flow was decided upon and written, hooking button and led behaviour together with sound effects and music