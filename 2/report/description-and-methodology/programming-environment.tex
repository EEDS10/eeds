\subsection{JTAGICE}
    See \cite{tdt4258-1}, page 2-3. 

\subsection{GNU Debugger}
    Picking up where \cite{tdt4258-1} left off, our heroes have since then discovered \texttt{tui} mode (text user interface mode), which while looking awfully nice breaks the Makefile. Likely due to \texttt{-tui} causing gdb to capture input differently, and the Makefile relies on a hack using cat and pipes.

    \texttt{avr32gdbproxy -f0,8Mb -a 0.0.0.0:1024} and \texttt{target extended-remote:1024}. While not 100 \% sure of what this does, it allows the program being debugged to be run again without starting gdb anew.


\subsection{Make}
    See \cite{tdt4258-1}, page 3.

\subsection{Other tools}
    \begin{itemize}
        \item OpenMPT, a MOD tracker, was used to examine MOD files during the development of libmodam.
        \item \texttt{vim} was employed as the authors' text editor of choice.
        \item \texttt{git} was used for version control.
        \item The project was hosted in a private GitHub repository.
        \item The report was written with \LaTeX.
        \item AVR32-specific flavors of GNU's \texttt{as} and \texttt{ld} were used to assemble and link executables.
        \item \texttt{avr32program} was used to program the STK1000 with the JTAGICE.
        \item The sound effects were recorded using Audacity on an ASUS UL30JT laptop.
    \end{itemize}