\section{Energy Efficiency}

Energy efficiency in computing is ever-important.
To save energy, the CPU could be set to sleep when no sounds are playing.
If this is done, the clock powering the ABDAC must be turned off before sleeping, so that it does not wake the CPU at once.
This is not done in our solution, as the the board makes a loud and ugly popping noise when it is switched on or off.
This popping noise is detrimental to the user experience, and ruins the functionality of the program.


\section{Testing}
We loaded up the code and pushed a button and omg it worked because it played sounds.
Also that one test where we just loaded the code to play a sound and it also worked.
The test where we recorded sound effects to see if the ADSR worked as expected.
The test where we triggered led changes from the abdac\_isr to measure oscillator speed (learned that div wasn't working how we thought it did, led to a hack, then finally figuring out a solution).


\section{Discussion}



\subsection{Ideas for improvement}

There are a number of measures that can be undertaken to improve the presented solution program.
The energy efficiency can be improved, as noted in the Energy Efficiency section of this report.
CPU sleeping and clock disabling are the most obvious measures, but they need to be done in a way that does not create nasty popping noises. 
Further, the clock rate of the CPU can be lowered to save energy, but this probably requires further optimization of the solution program.

To reduce the amount of noise from the board, the external DAC can be used.

To offload the CPU, the ABDAC can be set up to get samples using DMA.

To improve MOD compatability, more MOD effects can be implemented.

To increase the diversity and range of the sound effect synth, more controllable features can be implemented, such as volume slides and filters.
Additionally, more generator primitives (sine wave, triangle wave, pulse wave etc) can be implemented.
Finally, polyphony in the sound effect synth can greatly increase the number of possibilities.
