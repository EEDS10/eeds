
The MOD file format is an old music tracker file format originally created for the Commodore Amiga, a series of computers from the late eighties. TODO: IMAGE OF AN AMIGA, YO
The file format is tightly optimized for playback on the Amiga's audio hardware, so to understand the inner workings of the MOD file format, one should first know a little about how the Amiga's audio hardware works.

The Amiga's sound chip, called Paula, is capable of powering four simultaneous DMA-driven 8-bit PCM sample sound channels.
Each of these channels can be independently set to different sample frequencies many times per second.
The MOD format exploits this - it supports 4 simultaneous channels of sample playback, using the frequency modulation to change the pitch of the samples played in the different channels.

Internally, the music in a MOD file is stored as a set of PCM-coded predefined sounds, as well as a large table of note patterns containing information about which sounds should be played at which frequencies and at which time.
The MOD format also includes a large set of musical effects such as tremolo, vibrato, arpeggio, portamento and so on, a subset of which are implemented in the presented solution program.

The MOD file format is not a defined standard, and does therefore not have a formal specification.
The MOD format grew organically from the early Amiga demoscene in the eighties, so many different variants exist, each with with their own specialities and quirks.
The MOD Player presented in the solution is tailored to read so-called 'M.K.' MODs, generated by a MOD creator program ("tracker") called ProTracker.
These MOD files are called 'M.K.' MODs because they contain the magic number 'M.K.' in the file header.
This is one of the most popular MOD formats, and has become a sort informal standard amongst MOD trackers.

'M.K.' MODs can have a maximum of 31 bundled PCM-coded sounds, 128 patterns, each with 64 note divisions for each of the four channels, and a a 128 item long list of which patterns should be played in what order.

Image XXX shows an image of a MOD file being edited in a tracker program.
Each column represents one channel, and each row represents one of the 64 divisions of a pattern.
The currently played division is traditionally kept vertically centered in the middle of the screen, as in this image.




IMAGE: Protracker.png
Label: A four-channel MOD being played in ProTracker. Image courtesy of Alec Graggamoor.


Herer are the resources that were used by the way:
/* http://www.mediatel.lu/workshop/audio/fileformat/h_mod.html */
/* http://archive.cs.uu.nl/pub/MIDI/DOC/MOD-info */
/* https://bel.fi/alankila/modguide/ */
/* http://16-bits.org/mod/ */
