In order to write a program to generate sound, one should first study the physical properties of sound, and research different strategies to generate sound in a digital environment.

``Sound is a mechanical wave that is an oscillation of pressure composed of frequencies within the range of hearing.''\footnote{http://en.wikipedia.org/wiki/Sound} Humans can percieve sounds with frequencies that range from about 20Hz - the lowest of basses - to about 20kHz - the highest of high-pitched whining\footnote{http://en.wikipedia.org/wiki/Audio_frequency}.
Sound is inherently analog, and requires some form of digital representation to be able to be generated by the AP7000, which is a digital device.
If a sound wave is regarded as a continuous signal representing wave amplitude with respect to time, one straight-forward way of representing a sound wave digitally is to simply have a list of integer signal samples at a fixed, preferrably small, time interval.
The integers represent the amplitude of the sound wave at the given time.
This format is called is known as PCM (Pulse Code Modulation) and is the format the AP7000 expects for its digital-to-analog converter.

There are different strategies available for preparing the stream of integers that need to be sent to the digital-to-analog converter to generate a sound.
One strategy is simply to store the prepared list of integers somewhere in memory, and then copy it over to the DAC integer by integer as they are consumed.
This strategy is analogous to rasterized bit maps in the image processing world.
While being easy to implement and capable of representing all kinds of sounds, this strategy requires a great deal of memory (integer size $\times$ sample rate of bytes per second).
As an example, a three minute long song stored at 16 bits per sample at a sample rate of 22050Hz, requires $16$ bits/sample $\times 180$ seconds $\times 180$ seconds $\times 22050$ samples/second $\approx 7.57$ MiB.
This is not a viable strategy when working with low memory platforms like the STK1000 (the AP7000's flash memory can store 8 MiB).

Another strategy is the \emph{generative approach}.
This strategy is analogous to vector based images in the image processing world.
The idea is to generate samples at run-time based on configurations read from memory, rather than reading the pre-generated values from memory.
This is a more CPU-hungry approach, but requires less memory than the previous strategy.
This strategy is used for the sound effect synth in the presented solution program.

A third strategy is a \emph{hybrid approach}, where small sample lists are pre-bundled with the program, and generative rules are used to play back the samples at different times with different parameters.
This is the approach used in the music player in the presented solution program.
