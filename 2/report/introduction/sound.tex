In order to write a program to generate sound, one should first study the physical properties of sound, and research different strategies to generate sound in a digital environment.

Sound is a mechanical wave that is an oscillation of pressure composed of frequencies within the range of hearing\footnote{http://en.wikipedia.org/wiki/Sound}. Humans can percieve sounds with frequencies that range from about 20Hz - the lowest of basses - to about 20kHz - the highest of high-pitched whining\footnote{http://en.wikipedia.org/wiki/Audio_frequency}.
Sound is inherently analog, and requires some form of digital representation to be able to be generated by the AP7000, which is a digital device.
Regarding a sound wave as a continuous signal representing wave amplitude with respect to time, one straight-forward way of representing a sound wave digitally is to simply have an list of integer signal samples at a fixed, preferrably small, time interval.
This format is called PCM, or Pulse Code Modulation, and is indeed the format the AP7000 expects for its digital-to-analog converter.

There are different strategies available for preparing the stream of integers that needs to be sent to the digital-to-analog converter to generate a sound.
One strategy is simply to store the prepared list of integers somewhere in memory, and then copy it over to the DAC integer by integer as they are consumed.
This strategy is analogous to rasterized bit maps in the image world.
This strategy, while easy to implement, and is able to represent all kinds of sounds, requires a great deal of memory (integer size * sample rate of bytes per second, in fact).
As an example, a three minute song, when stored at 16 bits per sample at a generously low sample rate of 22050Hz, requires 16 bits/sample * 180 seconds * 22050 samples/second = ̃ca 7.57 megabytes.
To put this in perspective, the entire flash memory of the AP7000 is 8 megabytes.
On a low memory platform like the STK1000, this is therefore not a great strategy.

Another strategy is the generative approach.
This strategy is analogous to vector based images in the image world.
The idea is to generate samples at run-time based on configurations read from memory, rather than reading the pre-generated values from memory.
This is a more CPU-hungry approach, but requires less memory than the previous strategy.
This strategy is used for the sound effect synth in the presented solution program.

A third strategy is a hybrid approach, where small sample lists are pre-bundled with the program, and generative rules are used to play back the samples at different times with different parameters.
This is the approach used in the music player in the presented solution program.
