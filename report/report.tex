
\documentclass{article}
%%%%%%%%%%%%%%%%%%%%%%%%%%%%%%%%%%%%%%%%%%%%%%%%%%%%%%%%%%%%%%%%%%%%%%%%%%%%%%%%%%%%%%%%%%%%%%%%%%%%%%%%%%%%%%%%%%%%%%%%%%%%%%%%%%%%%%%%%%%%%%%%%%%%%%%%%%%%%%%%%%%%%%%%%%%%%%%%%%%%%%%%%%%%%%%%%%%%%%%%%%%%%%%%%%%%%%%%%%%%%%%%%%%%%%%%%%%%%%%%%%%%%%%%%%%%
\usepackage{geometry}
\usepackage{fancyhdr}
\usepackage[pdftex]{graphicx}

%TCIDATA{OutputFilter=LATEX.DLL}
%TCIDATA{Version=5.50.0.2953}
%TCIDATA{<META NAME="SaveForMode" CONTENT="1">}
%TCIDATA{BibliographyScheme=Manual}
%TCIDATA{Created=Monday, January 30, 2012 17:20:46}
%TCIDATA{LastRevised=Monday, February 27, 2012 12:06:08}
%TCIDATA{<META NAME="GraphicsSave" CONTENT="32">}
%TCIDATA{<META NAME="DocumentShell" CONTENT="Standard LaTeX\Blank - Standard LaTeX Article">}
%TCIDATA{CSTFile=40 LaTeX article.cst}

\newenvironment{proof}[1][Proof]{\noindent\textbf{#1.} }{\ \rule{0.5em}{0.5em}}

% Sets page margins to 1", which is standard
\geometry{left=1in,right=1in,top=1in,bottom=1in} 

% allows the included extensions of graphic files
\DeclareGraphicsExtensions{.pdf,.png,.jpg}

% sets/adds graphic path. If empty it just looks around the folder the .tex file is in
\graphicspath{{}}

% I do not remember what this does
\setlength{\headheight}{15.2pt}

% allows the xhead parameters
\pagestyle{fancy}

% Sets the left header
\lhead{Group \#10}

% Sets the right header
\rhead{Group 10 IS BEST GROUP}

% ????
% \input{tcilatex}

% everything before this is considered the header or whatever.
\begin{document}

% INCLUDEGRAPHICS EXPLANATION
% \includegraphics[scale=1]{name of file}
% sometimes you want to twice encase the filename in squiggly brackets. I do not know why but sometimes it is required.

% begin title page, use \\ for newline
\title{\#EEDS\\Report from Lab Assignment \#1\\TDT4258 Energy Efficient Computer Systems}

% now one can list the authors, \textbf{} makes bold text
\author{Emil Taylor Bye}
\author{Sigve Sebastian Farstad}
\author{Odd M. Trondrud}

% make title page
\maketitle

% create new page
\newpage

% uncomment this if you want to include a ToC
% \tableofcontents
% disables chapter, section and subsection numbering
% set to 1 or higher to enable section numbering
% it goes something like 0:Part, 1:Section, 2:Section, 3:Subsection, 4:Subsubsection, ...
\setcounter{secnumdepth}{-1}

% use \newpage to get newpage
% \bigskip to get some space between paragraphs
% \texttt{TEXTHERE} to get typewriter-like (basically just monospaced) text

\part{Abstract}

This report presents our solution to assignment #1 of TDT4258 at NTNU, spring 2013.
In the assigment, an Atmel STK1000 development board is programmed to display a moveable LED 'paddle' using AVR32 assembly and the GNU tool chain.

\part{Introduction}

The objective of this lab assignment was to make a program for the STK1000 which allowed the user to control which of the LED diodes on a row of diodes should light up.
This was to be accomplished by moving it left or right by pressing the appropriate buttons.


\part{Description and Methodology}

This is the description and methodology.

\subsection{Configuration of the STK1000}

\subsubsection{Jumpers}

The STK1000 has (number) jumpers that can be set to configure the board.
For this assignment, the following settings were set for the jumpers:

\begin{table}
\begin{tabular}{|l|l|}

SW1 &
Set to SPI0.
\\

SW2 &
Set to PS2A/MMCI/USART1.
\\

SW3 &
Set to SSC0/PWM[0,1]/GCLK.
\\

SW4 &
Set to GPIO.
\\

SW5 &
Set to LCDC.
\\

SW6 &
Set to GPIO.
\\

JP4 &
Set to "INT. DAC".
\\

JP5 &
Set to "INT. DAC".
\\

\end{tabular}
\end{table}


\subsubsection{GPIO connections}

The STK1000 provides a general purpose input/output (GPIO) interface.
16 of the available 32 signal lines are in this assigment connected to the 8 on-board LEDs and the 8 on-board switches, to be used as output and input for the paddle move program.

The buttons were connected to GPIO0-GPIO7 (J1 on the STK1000) using a flat cable as in <diagram X>, mapping them to ports 0-7 of PIOB.
The choice of low port numbers 0-7 is convenient for coding later, and the choice of PIOB is purely memnonic ('B' for buttons).

The leds were connected to GPIO16-GPIO23 (J3 on the STK1000) using a flat cable as in <diagram X>, makking them to ports 0-7 of PIOB.
Having the same port numbers for the buttons and the LEDs is a nice convenience for cleaner code.


<diagram X>, note the orientation of the flat cables


\subsection{Programming environment}

\subsubsection{JTAGICE}

The AP7000 sisterboard on the STK1000 provides a JTAG interface which is used for programming and debugging of the board.
The development PC was connected to the JTAG interface of the STK1000 using an Atmel JTAGICE mk II (firmware 7.29).
The JTAGICE does not require external power as long as it is connected to the PC over USB.


\subsubsection{GNU Debugger}

\subsubsection{Make, other tools etc}

\subsection{Development of the program}

\subsubsection{Setting up the LED diodes}

\subsubsection{Setting up the buttons}

\subsubsection{Interrupt Routine}

\subsubsection{Refactoring and Modularization}

\subsubsection{Register Overview}

<description of which registers do what>

\subsubsection{Program flow}

<diagrams, descriptions>

yuml.me-source for diagrams

Main program flow:
(start)->(Init)->(Main loop)->(Main loop)


Init:
(start)->(Load pointers)->(Set up start values in registers)->(Enable I/O with interrupts)->(end)

Main loop:
(start)->(Set leds)->(Sleep)[interrupt]-><interrupt routine>->[Was SW0 pressed][Yes]->(Move paddle right)->(end),[Was SW0 pressed][No]->(Move paddle left)->(end)


Set leds:
(start)->(Turn off all LEDs)->(Turn on the paddle LED)->(end)

Button interrupt routine:
(start)->(Read button states)->(Software debounce)->(Notify that the interrupt has been handled)->(end)

Debounce:
(start)->(Set register to a high constant)->(Is value in register equal to 0)[Yes]->(end),(Is value in register equal to 0)[No]->(Decrease value in register by 1)->(Is value in register equal to 0)

Read buttons:
(start)->(Read button status)->(Mask away buttons that were pressed in previous interrupt)->(end)

Move paddle right:
(start)->(Is paddle at far-right end of the board)[Yes]->(Move paddle to far-left)->(end),(Is paddle at far-right end of the board)[No]->(Move paddle one step to the left)->(end)

Move paddle left (analogous to move paddle right, but included for completeness):
(start)->(Is paddle at far-left end of the board)[Yes]->(Move paddle to far-right)->(end),(Is paddle at far-left end of the board)[No]->(Move paddle one step to the right)->(end)





\part{Results and Tests}


This is the results and tests section.

\subsection{Energy Efficiency}

TODO: we should get an amperemeter and measure actual efficiency of our programme with busy loops, sleep etc

\subsection{Testing}

	All tests assume that you are in possession of at least one (1) functional STK1000 development board (with cables), a JTAGICE mkII (with USB cable) and a computer with software and hardware capable of interfacing with the JTAGICE.
\paragraph{Button Functionality Test}
This test aims to uncover if pushing Button 0 has the desired effect (the desired effect from pushing Button 0 is that the paddle moves one LED to the right.)
\\ Prerequisites:
\begin{itemize}
	\item One (1) finger
	\item Functional eyesight
\end{itemize}
Procedure:
\begin{enumerate}
	\item Upload the code to the board (e.g using \texttt{make upload}).
	\item Push the board's reset button.
	\item Note the paddle's position.
	\item Push Button 0.
	\item Note the paddle's position.
\end{enumerate}
If the paddle's position in the last step is not one to the right of its position in step \#3, Button 0 does not have the appropriate functionality.
The test can be refactored to test Button 2: simply push Button 2 instead in step \#4 and note that the test will have failed if the paddle's position in step \#5 is not one to the left of its position in step \#3.

\paragraph{Measurement of Power Consumption}
This test measures the board's power consumption by removing a jumper and connecting an amperemeter in series.
Power consumption is traditionally preferably measured while the board is turned on and 
\\
Prerequisites:
\begin{itemize}
	\item{Multimeter or amperemeter}
	\item{Hands, or other tool with similar prehensile ability}
    \item{STK1000 development board with the desired program code running.}
\end{itemize}
Procedure:
\begin{enumerate}
    \item{Remove the jumper from the pins that are to be measured.}
    \item{Touch the pins that are now free from the jumper with the measuring device's probes.}
	\item{Press the \texttt{RESET} button on the STK1000.}
    \item{Read the measurement from the display of the measurement device.}
\end{enumerate}
The board can optionally be interacted with while performing the last step of the procedure in order to measure power consumption during certain interactions.



\part{Evaluation of Assignment}

Writing the lab-reports using the required design (Abstract, introduction, description and methodology, results and tests, evaluation of assignment, conclusion, references) felt a bit weird at first, probably because we did not read the supplied material regarding lab reports.

The assignment/compendium could include some questions for the students to reflect over, to push them in the right direction regarding what's important to include in the report.
As it is now, there aren't that many relevant tests (in hindsight, at least) beyond ``does it work per the assignment's specification?''.

All in all though we had fun and the experience was pretty enjoyable.

\part{Conclusion}

The finished video game is pretty smooth and will surely achieve a perfect score of 100/100 on Metacritic within the first twenty four hours of its release.

The authors feel they have gotten pretty good at cranking out reports and that it can be difficult to maintain energy efficiency as the number one priority when developing a video game.

Here is a video demonstration of the finished game: VIDEO LINK HERE????

\part{Acknowledgement}

Here, we acknowledge those who deserve to be acknowledged.


\part{References}

\input{references/references.tex}
\end{document}
