Before interrupts can be utilized we have to configure the board in an appropriate manner.
This process is outlined in section 2.5 of \cite{lab-compendium}.
Since we connected the buttons to PIO port B, we will be detailing how we enabled interrupts from PIO port B.
First, we set the appropriate bits in the Interrupt Enable Register to 1.
\code{84}{84}{The base address of PIO port B is loaded into r2.}
\code{97}{97}{The masks of Button 0 and Button 2 are loaded into r5}
\code{114}{115}{Interrupts are enabled for Button 0 and Button 2}
Before we enable interrupts for Button 0 and Button 2, we defensively disable interrupts for everything by loading \texttt{0xff} into the Interrupt Disable Register.
Finally, we enable interrupts globally by setting the bit GM (Global Interrupt Mas) in the processor's status register to 0. % taken from pg 26 of the compendium
\code{126}{127}{}
% EVBA = Exception Vector Base Address