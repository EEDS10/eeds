Before interrupts can be utilized we have to configure the board in an appropriate manner.
\subsubsubsection{Enabling Interrupts}
This process is outlined in section 2.5 of \cite{lab-compendium}.
Since we connected the buttons to PIO port B, we will be detailing how we enabled interrupts from PIO port B.
First, we set the appropriate bits in the Interrupt Enable Register to 1.
\code{84}{84}{The base address of PIO port B is loaded into r2.}
\code{97}{97}{The masks of Button 0 and Button 2 are loaded into r5}
\code{114}{115}{Interrupts are enabled for Button 0 and Button 2}
Before we enable interrupts for Button 0 and Button 2, we defensively disable interrupts for everything by loading \texttt{0xff} into the Interrupt Disable Register.
Finally, then enable interrupts globally by setting the bit GM (Global Interrupt Mask) in the processor's status register to 0. % taken from pg 26 of the compendium
\code{126}{127}{}
% EVBA = Exception Vector Base Address
\subsubsubsection{Loading the interrupt routine}
After enabling interrupts, we have to inform the processor about what to do when it receives an interrupt.
First, we specify the address of our interrupt routine to be used as the autovector when the interrupt controller receives an interrupt from group 14.
\code{86}{86}{The base address of the interrupt controller is loaded into r7.}
\code{87}{87}{The address of our interrupt routine is loaded into r8.}
\code{120}{121}{The address of our interrupt routine is stored in IPR14's register in the interrupt controller.}
Then we have to set the processor's EVBA\footnote{Exception Vector Base Address} register to the desired value, i.e. zero.
\code{117}{118}{Setting the EVBA to 0. 0 is stored in r1.}
This is because the interrupt routine address is calculated by performing bitwise logical AND on the EVBA and autovector, where the autovector is represented by the 14 least significant bits in the interrupt routine address. By setting the EVBA to 0, the interrupt routine address simply becomes the autovector, which we have specified as the address of our interrupt routine.