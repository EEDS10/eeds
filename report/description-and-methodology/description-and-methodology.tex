This section describes how the paddle program was developed.
It covers setup and configuration, tools and program details.

\subsection{Configuration of the STK1000}

    \subsubsection{Jumpers}

        The STK1000 has (number) jumpers that can be set to configure the board.
For this assignment, the following settings were set for the jumpers:

\begin{table}
\begin{tabular}{|l|l|}

SW1 &
Set to SPI0.
\\

SW2 &
Set to PS2A/MMCI/USART1.
\\

SW3 &
Set to SSC0/PWM[0,1]/GCLK.
\\

SW4 &
Set to GPIO.
\\

SW5 &
Set to LCDC.
\\

SW6 &
Set to GPIO.
\\

JP4 &
Set to "INT. DAC".
\\

JP5 &
Set to "INT. DAC".
\\

\end{tabular}
\end{table}


    \subsubsection{GPIO connections}

        The STK1000 provides a general purpose input/output (GPIO) interface.
16 of the available 32 signal lines are in this assigment connected to the 8 on-board LEDs and the 8 on-board switches, to be used as output and input for the paddle move program.

The buttons were connected to GPIO0-GPIO7 (J1 on the STK1000) using a flat cable as in <diagram X>, mapping them to ports 0-7 of PIOB.
The choice of low port numbers 0-7 is convenient for coding later, and the choice of PIOB is purely memnonic ('B' for buttons).

The leds were connected to GPIO16-GPIO23 (J3 on the STK1000) using a flat cable as in <diagram X>, makking them to ports 0-7 of PIOB.
Having the same port numbers for the buttons and the LEDs is a nice convenience for cleaner code.


<diagram X>, note the orientation of the flat cables


\subsection{Programming environment}

    \subsubsection{JTAGICE}

        The AP7000 sisterboard on the STK1000 provides a JTAG interface which is used for programming and debugging of the board.
The development PC was connected to the JTAG interface of the STK1000 using an Atmel JTAGICE mk II (firmware 7.29).
The JTAGICE does not require external power as long as it is connected to the PC over USB.


    \subsubsection{GNU Debugger}
    % Done.
        The instructions presented in the compendium were followed in an attempt to employ the GNU debugger, but the proxy connection could not be established.
Because of this, the GNU debugger did not play a important role in the development of the solution.
After the main development of the program was complete, the proper setup procedure for the debugger was discovered by another group and subsequently shared.
For completeness, the already developed program was debugged using the GNU debugger, to confirm that debugging did indeed work.
As the setup procedure differs somewhat from that of the compendium, it is reproduced in its entirety here.
TODO: reproduce setup procedure here.
TODO: run up to the lab and use gdb a bit, so we can write about it here afterwards.


    \subsubsection{Make, other tools etc}

    Make, vim, git.

\subsection{Development of the program}

This section details the steps taken during the development of the program.
Initially, a bare-bones program was developed with minimal functionality, to get familiar with the environment.
Features were added iteratively, starting with simple LED and button integration, and moving on to more sophisticated interrupt-oriented logic/program flow.

    \subsubsection{Setting up the LEDs}
        
        On the STK1000, the connection to the output LEDs must be set up before the LEDs can be used in a program.
First, the I/O pins that the LEDs are connected to must be enabled.
In this assignment, the LEDs were connected to the pins \texttt{GPIO16-23}, corresponding to \texttt{PIO C} lines \texttt{0-7}.
To enable the correct I/O pins, we must therefore set to 'high' bits \texttt{0-7} of the \texttt{PIO C} PIO Enable Register (\texttt{PIOC PER}), as in listing 1.
Here, \texttt{r3} is the base offset of \texttt{PIOC}, \texttt{AVR32\_PIO\_PER} is the PIO Enable Register offset, and \texttt{r6} contains the bit field indicating which pins to enable.

\code{99}{100}{Enable the I/O pins}

Second, the I/O pins must be set to act as output pins, as opposed to input pins.
This is done by setting to 'high' the corresponding bits (\texttt{0-7}) of the \texttt{PIO C} Output Enable Register (\texttt{PIOC OER}), as in listing 2.
Here, \texttt{r3} is the base offset of \texttt{PIOC}, \texttt{AVR32\_PIO\_OER} is the Output Enable Register offset, and \texttt{r6} contains the bit field indicating which pins to set as outputs.

\code{102}{103}{Set the pins to act as output pins}

Once this is done, LEDs can be turned on by writing the appropriate bits to \texttt{PIO C} Set Output Data Register (\texttt{PIOC SODR}), as in listing 3.
Here, \texttt{r3} is the offset of \texttt{PIOC}, \texttt{AVR32\_PIO\_SODR} is the Set Output Data Register offset, and \texttt{r4} contains the bit field indicating which LEDs to switch on.

\code{176}{177}{Switch on LEDs}

Analogously, LEDs can be turned of by writing the approtiate bits to \texttt{PIO C} Clear Output Data Register (\texttt{PIOC CODR}), as in listing 4. 
Here, \texttt{r3} is the offset of \texttt{PIOC}, \texttt{AVR32\_PIO\_SODR} is the Set Output Data Register offset, and \texttt{r4} contains the bit field indicating which LEDs to switch off.

\code{173}{174}{Switch off LEDs}


    \subsubsection{Setting up the buttons}

        The connection to the input buttons must be set up before the buttons can be used in a program.
First, the I/O pins that the buttons are connected to must be enabled.
In this assignment, the buttons are connected to the pins GPIO0-7, corresponding to PIO B lines 0-7.
To enable the correct I/O pins, we must therefore set to 'high' bits 0-7 of the PIO B PIO Enable Register (PIOB PER), as in listing X.
Here, r2 is the base offset of PIOB, AVR32\_PIO\_PER is the PIO Enable Register offset, and r6 contains the bitfield indicating which pins to enable.

\code{105}{106}{Enabling I/0 pins}

Second, the pull-up resistors for the buttons must be enabled. This is because <reason>.
This is done by setting to 'high' the corresponding bits (0-7) of the PIO B Pull-Up Enable Register (PIOC PUER), as in listing x.
Here, r2 is the base offset of PIOB, AVR32\_PIO\_PUER is the Pull-Up Enable Register offset, and r6 contains the bitfield indicating which pull-up resistors should be enabled.

\code{108}{109}{Enabling pull-up resistors}

Once this is done, the button state can be read by reading the appropriate bits from PIO B Pin-Data Status Register (PIOB PDSR), as in listing x.
Here, r2 is the offset of PIOB, AVR32\_PIO\_PDSR is the Pin-Data Status Register offset.

\code{225}{226}{Reading the button state}




    \subsubsection{Setting up the interrupts}

        Before interrupts can be utilized we have to configure the board in an appropriate manner.
This process is outlined in section 2.5 of \cite{lab-compendium}.
Since we connected the buttons to PIO port B, we will be detailing how we enabled interrupts from PIO port B.
First, we set the appropriate bits in the Interrupt Enable Register to 1.
\code{84}{84}{The base address of PIO port B is loaded into r2.}
\code{97}{97}{The masks of Button 0 and Button 2 are loaded into r5}
\code{114}{115}{Interrupts are enabled for Button 0 and Button 2}
Before we enable interrupts for Button 0 and Button 2, we defensively disable interrupts for everything by loading \texttt{0xff} into the Interrupt Disable Register.
Finally, we enable interrupts globally by setting the bit GM (Global Interrupt Mas) in the processor's status register to 0. % taken from pg 26 of the compendium
\code{126}{127}{}
% EVBA = Exception Vector Base Address

    \subsubsection{Interrupt Routine}
    % bein' worked at
        %% EXECUTIVE SUMMARY
Our interrupt routine first reads the state of the buttons before calling a debounce routine which runs for some amount of time.
When the debounce routine is finished the button interrupt routine notifies that the interrupt has been handled before returning to the main loop.
\\ % starts a new paragraph
The debounce routine's running time can be modified by altering the DEBOUNCE constant, which can be considered the routine's loop counter. The routine keeps the CPU busy by repeatedly subtracting one from this value until it reaches zero. This prevents further interrupts from being registered for the duration of the debouncing routine.

%% full blown deets.
In order to implement an interrupt routine we have to enable the relevant interrupts and tell the interrupt controller where it can find the interrupt routine.
Interrupts are enabled for specific I/O units by loading the appropriate values into the Interrupt Enable Register.
We want to enable interrupts for SW0 and SW2.
Their masks are \texttt{0x1} and \texttt{0x4}, respectively.


We enable interrupts by loading the appropriate values into the Interrupt Enable Register (PER).
Specifically, the \"appropriate values\" are the masks for SW0 and SW2, which are \texttt{0x1} and \texttt{0x4}, respectively.
Since our buttons are connected to PIOB, we load the masks
More specifically, we load these masks into a register, and then 
\\*
We load PIO B's base address into r2
\texttt{
		/* */
		lddpc r2, piob_offset
		/* The masks of SW_0 and SW_2 are loaded into r5 */
		mov r5, SW_0 | SW_2
		/* */
		st.w r2[AVR32_PIO_IER], r5
}
% store the mask of SW0 and SW2 in r5

% starts a new line, but not a new paragraph
% PIO_IER is the interrupt enable register
\texttt{st.w r2[AVR32_PIO_IER], r5 }

This is done by storing the appropriate values in the PIO Enable Register (PER).
We enabled button interrupts for \texttt{SW0} and \texttt{SW2} by loading their masks into the PIOB, with the appropriate offset to specify
GPIO pins = PIO
general purpose input/output pins

% load the pointer to PIOB's base address into r2
lddpc r2, piob_offset
% load the pointer to the interrupt controller's base address into r7
lddpc r7, intc_base
% store the address of the interrupt routine in r8
mov r8, button_interrupt_routine
% store the mask of SW0 and SW2 in r5
mov r5, SW_0 | SW_2
% enable button interrupts for SW0 and SW2
% PIO_IER is the interrupt enable register
st.w r2[AVR32_PIO_IER], r5 
% stores the address of our button interrupt routine in the interrupt controller
st.w r7[AVR32_INTC_IPR14], r8

\texttt{st.w r2[AVR32_PIO_IER], r5}
Enables button interrupts for SW0 and SW2.
First, interrupts must be disabled


lddpc r7, intc_base
% intc_base is a pointer to the address of the interrupt controller

mov r8, button_interrupt_routine
% the address of the interrupt routine is loaded into r8

st.w r7[AVR32_INTC_IPR14], r8
% the address of the interrupt routine is stored in the interrupt controller's button h

The time is specified 
First the state of the buttons is read.
The debounce routine simply waits for a time decided by the DEBOUNCE constant in CONFIGURATION.

st.w rd, rs
% writes the value in rs to the address found in rd


et spesielt minneområde som skal inneholde adressen til en rutine som vil bli kjørt når kortet mottar en relevant interrupt. I dette tilfelle er relevante interrupts knappe-interrupts.

rcall read_buttons
rcall debounce

ld.w r0, r2[AVR32_PIO_ISR]

rd, rs

rete


    \subsubsection{Refactoring and Modularization}

        YO!

    \subsubsection{Program flow}

        <diagrams, descriptions>

yuml.me-source for diagrams

Main program flow:
(start)->(Init)->(Main loop)->(Main loop)


Init:
(start)->(Load pointers)->(Set up start values in registers)->(Enable I/O with interrupts)->(end)

Main loop:
(start)->(Set leds)->(Sleep)[interrupt]-><interrupt routine>->[Was SW0 pressed][Yes]->(Move paddle right)->(end),[Was SW0 pressed][No]->(Move paddle left)->(end)


Set leds:
(start)->(Turn off all LEDs)->(Turn on the paddle LED)->(end)

Button interrupt routine:
(start)->(Read button states)->(Software debounce)->(Notify that the interrupt has been handled)->(end)

Debounce:
(start)->(Set register to a high constant)->(Is value in register equal to 0)[Yes]->(end),(Is value in register equal to 0)[No]->(Decrease value in register by 1)->(Is value in register equal to 0)

Read buttons:
(start)->(Read button status)->(Mask away buttons that were pressed in previous interrupt)->(end)

Move paddle right:
(start)->(Is paddle at far-right end of the board)[Yes]->(Move paddle to far-left)->(end),(Is paddle at far-right end of the board)[No]->(Move paddle one step to the left)->(end)

Move paddle left (analogous to move paddle right, but included for completeness):
(start)->(Is paddle at far-left end of the board)[Yes]->(Move paddle to far-right)->(end),(Is paddle at far-left end of the board)[No]->(Move paddle one step to the right)->(end)



    \subsubsection{Register Overview}

        The STK1000 has 13 general purpose registers named r0-r12. The programmer is free to use these registers for whatever they want.
However, several conventions are commonplace to introduce a certain degree of structure.

Conventions:

r0 is a scratch register, used for intermediate calculations and such.
r1 holds the constant 0.

...

r8, r9, r10, r11, r12:

used to hold parameters when calling a routine.

r12:

used to hold the return value when returning from a routine.


List of registers and what they are used for in the paddle move program:

\begin{table}
    \begin{tabular}{|l|l|}
        r0  & Scratch register used to hold intermediate values. \\
        r1  & Constant: 0 \\ 
        r2  & Constant: base offset to PIOB \\ 
        r3  & Constant: base offset to PIOC \\ 
        r4  & Variable: holds the position of the paddle \\ 
        r5  & Constant: 5 \\ 
        r6  & Constant: 0xff \\ 
        r7  & Variable: holds the previous button state \\ 
        r8  & Constant: pointer to button interrupt routine \\ 
        r9  & (not used) \\ 
        r10 & (not used) \\ 
        r11 & (not used) \\ 
        r12 & Holds return value from routines \\
    \end{tabular}
\end{table}


