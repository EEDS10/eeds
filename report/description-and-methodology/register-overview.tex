The STK1000 has 13 general purpose registers named r0-r12. The programmer is free to use these registers for whatever they want.
However, several conventions are commonplace to introduce a certain degree of structure.

Conventions:

r0 is a scratch register, used for intermediate calculations and such.
r1 holds the constant 0.

...

r8, r9, r10, r11, r12:

used to hold parameters when calling a routine.

r12:

used to hold the return value when returning from a routine.


For a list of registers and what they are used for in the paddle move program see table \ref{register-table}.

\begin{table}
    \centering
    \begin{tabular}{|l|l|}
        \hline
        register  & Usage \\
        \hline
        \hline
        r0  & Scratch register used to hold intermediate values. \\
        \hline
        r1  & Constant: 0 \\ 
        \hline
        r2  & Constant: base offset to PIOB \\ 
        \hline
        r3  & Constant: base offset to PIOC \\ 
        \hline
        r4  & Variable: holds the position of the paddle \\ 
        \hline
        r5  & Constant: 5 \\ 
        \hline
        r6  & Constant: 0xff \\ 
        \hline
        r7  & Variable: holds the previous button state \\ 
        \hline
        r8  & Constant: pointer to button interrupt routine \\ 
        \hline
        r9  & (not used) \\ 
        \hline
        r10 & (not used) \\ 
        \hline
        r11 & (not used) \\ 
        \hline
        r12 & Holds return value from routines \\
        \hline
    \end{tabular}
    \caption{Overview over register use}
    \label{register-table}
\end{table}
