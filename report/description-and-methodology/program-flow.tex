This section presents a detailed overview of the program flow of the final interrupt-oriented paddle program.


yuml.me-source for diagrams

Main program flow:
(start)->(Init)->(Main loop)->(Main loop)


Init:
(start)->(Load pointers)->(Set up start values in registers)->(Enable I/O with interrupts)->(end)

Main loop:
(start)->(Set leds)->(Sleep)[interrupt]-><interrupt routine>->[Was SW0 pressed][Yes]->(Move paddle right)->(end),[Was SW0 pressed][No]->(Move paddle left)->(end)


Set leds:
(start)->(Turn off all LEDs)->(Turn on the paddle LED)->(end)

Button interrupt routine:
(start)->(Read button states)->(Software debounce)->(Notify that the interrupt has been handled)->(end)

Debounce:
(start)->(Set register to a high constant)->(Is value in register equal to 0)[Yes]->(end),(Is value in register equal to 0)[No]->(Decrease value in register by 1)->(Is value in register equal to 0)

Read buttons:
(start)->(Read button status)->(Mask away buttons that were pressed in previous interrupt)->(end)

Move paddle right:
(start)->(Is paddle at far-right end of the board)[Yes]->(Move paddle to far-left)->(end),(Is paddle at far-right end of the board)[No]->(Move paddle one step to the left)->(end)

Move paddle left (analogous to move paddle right, but included for completeness):
(start)->(Is paddle at far-left end of the board)[Yes]->(Move paddle to far-right)->(end),(Is paddle at far-left end of the board)[No]->(Move paddle one step to the right)->(end)

