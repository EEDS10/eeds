The connection to the output LEDs must be set up before the LEDs can be used in a program.
First, the I/O pins that the LEDs are connected to must be enabled.
In this assignment, the LEDs are connected to the pins GPIO16-23, corresponding to PIO C lines 0-7.
To enable the correct I/O pins, we must therefore set to 'high' bits 0-7 of the PIO C PIO Enable Register (PIOC PER), as in <code excerpt>.

    st.w r3[AVR32_PIO_PER], r6

    <code excerpt>: Here, r3 is the base offset of PIOC, AVR32_PIO_PER is the PIO Enable Register offset, and r6 contains the bitfield indicating which pins to enable.


Second, the I/O pins must be set to act as output pins, as opposed to input pins.
This is done by setting to 'high' the corresponding bits (0-7) of the PIO C Output Enable Register (PIOC OER), as in <code excerpt>.

    st.w r3[AVR32_PIO_OER], r6

    <code excerpt>: Here, r3 is the base offset of PIOC, AVR32_PIO_OER is the Output Enable Register offset, and r6 contains the bitfield indicating which pins to set as outputs.


Once this is done, LEDs can be turned on by writing the appropriate bits to PIO C Set Output Data Register (PIOC SODR), as in <code excerpt>.

    st.w r3[AVR32_PIO_SODR], r4
    
    <code excerpt>: Here, r3 is the offset of PIOC, AVR32_PIO_SODR is the Set Output Data Register offset, and r4 contains the bitfield indicating which LEDs to switch on.

Analogously, LEDs can be turned of by writing the approtiate bits to PIO C Clear Output Data Register (PIOC CODR), as in <code_excerpt>.

    st.w r3[AVR32_PIO_CODR], r6

    <code excerpt>: Here, r3 is the offset of PIOC, AVR32_PIO_SODR is the Set Output Data Register offset, and r4 contains the bitfield indicating which LEDs to switch off.
