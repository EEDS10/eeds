The connection to the input buttons must be set up before the buttons can be used in a program.
First, the I/O pins that the buttons are connected to must be enabled.
In this assignment, the buttons are connected to the pins GPIO0-7, corresponding to PIO B lines 0-7.
To enable the correct I/O pins, we must therefore set to 'high' bits 0-7 of the PIO B PIO Enable Register (PIOB PER), as in listing X.
Here, r2 is the base offset of PIOB, AVR32\_PIO\_PER is the PIO Enable Register offset, and r6 contains the bitfield indicating which pins to enable.

\code{105}{106}{Enabling I/0 pins}

Second, the pull-up resistors for the buttons must be enabled. This is because <reason>.
This is done by setting to 'high' the corresponding bits (0-7) of the PIO B Pull-Up Enable Register (PIOC PUER), as in listing x.
Here, r2 is the base offset of PIOB, AVR32\_PIO\_PUER is the Pull-Up Enable Register offset, and r6 contains the bitfield indicating which pull-up resistors should be enabled.

\code{108}{109}{Enabling pull-up resistors}

Once this is done, the button state can be read by reading the appropriate bits from PIO B Pin-Data Status Register (PIOB PDSR), as in listing x.
Here, r2 is the offset of PIOB, AVR32\_PIO\_PDSR is the Pin-Data Status Register offset.

\code{225}{226}{Reading the button state}


