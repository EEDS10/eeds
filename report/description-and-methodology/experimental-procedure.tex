The described procedures were carried out in the PC lab on the fourth floor of the IT-Vest building at NTNU, room 458.
The lab is stocked with development PCs and boxed AVR32STK1000 development boards with all necessary extra equipment, such as cables and a JTAGICE.
An AVR32STK1000 board was removed from its box and the jumpers were set to the appropriate positions (\cite{lab-compendium}, pg 37).
A computer was booted up and the unboxed AVR32STK1000 was connected to it using a USB JTAGICE.
The assembly program was then developed on the development PC.
Initially, a bare-bones program was developed with minimal functionality, to get familiar with the environment.
Features were added iteratively, starting with simple LED and button integration, and moving on to more sophisticated interrupt-oriented logic/program flow.
The development of the program is elaborated upon in section 4: Development of the program.

Although the GNU debugger would have been a useful tool during development, it was not used, as the GDB setup instructions in the compendium \cite{lab-compendium} were erroneous, and efforts to find a work-around did not prove to be successful.

During this development, the code was manually tested on the STK1000 by uploading the program using the JTAGICE.


Finally the current throughput over the board's various pins was measured while an interrupt-based program was running, and again with a busy-waiting program in order to compare the energy efficiency of the two solutions.
