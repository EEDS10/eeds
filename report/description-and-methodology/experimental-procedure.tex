The first thing we did was locate the lab, on the fourth floor of the IT-Vest building, room 458. 
We removed one of the AVR32STK1000 boards from their box and set the jumpers to the appropriate positions (\cite{lab-compendium}, pg 37).
We booted up one of the computers and connected the AVR32STK1000 to it. 
Our first piece of code simply enabled all the LEDs and turned them on. Once we got the LEDs working we enabled the buttons.
We followed this up by writing code to turn on just one of the LEDs, designated the ''paddle'' per the assignment's description (\cite{lab-compendium}, pg 37).
Shortly thereafter we gave up on trying to figure out GDB.
Code was written to move right when \texttt{$SW_0$} is pushed, and left when\texttt{$SW_2$} is pushed, employing arithmetic shift to move the paddle bit in the appropriate direction.
However, a single push of either button caused the paddle to disappear, due to a combination of us simply letting the paddle's bit overflow when it reached either edge and the \emph{bouncing effect} (\cite{lab-compendium}, pg 22). We altered our code so that the paddle would ''loop around'' to the opposite side of the row of LEDs when it is ''pushed off'' either side.
When pushing either buttons, all the LEDs would light up briefly in turn at such high a speed that it looked like they were all lit simultaneously, again due to the bouncing effect.
Implementing debouncing (\cite{lab-compendium}, Fig. 2.9a) seemed like the next logical step, so the next thing we did was change our program to an interrupt-based solution.
Then we implemented debouncing.
\\ % new line, paragraph
We had an issue where the paddle would move over one additional LED when either button was pushed (i.e. from \texttt{$LED_n$} to \texttt{$LED_n+2$} rather than from \texttt{$LED_n$} to \texttt{$LED_n+1$}), which we fixed by, in effect, by ignoring every second registered push of either buttons.
Finally we measured the current on the board's various pins while an interrupt-based program running, and again with a busy-waiting program in order to compare the energy efficiency of the two solutions.