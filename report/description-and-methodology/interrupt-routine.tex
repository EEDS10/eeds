For the interrupt routine to work, we need to set the processor's EVBA\footnote{Exception Vector Base Address} register to the desired value, i.e. the base address of our interrupt routine\footnote{For some reason, this value is 0}.
\code{117}{118}{Setting the EVBA to 0}


\code{86}{86}{The base address of the interrupt controller is loaded into r7}
\code{87}{87}{The address of our interrupt routine is loaded into r8}

EVBA is set to 0
and the address of our interrupt routine is stored in the interrupt controller (group 14)

Page 24
Do we set the EVBA to 0 because the PIO B controller has interrupt line number 0 (group4 is accounted for )
According to \cite{lab-compendium} (pg 24), 
We tell the interrupt controller 
%% EXECUTIVE SUMMARY
The button interrupt routine first reads the state of the buttons before calling a debounce routine which runs for some amount of time.
When the debounce routine is finished the button interrupt routine notifies that the interrupt has been handled before returning to the main loop.
\\ % starts a new paragraph
The debounce routine's running time can be modified by altering the DEBOUNCE constant, which can be considered the routine's loop counter. The routine keeps the CPU busy by repeatedly subtracting one from this value until it reaches zero. This prevents further interrupts from being registered for the duration of the debouncing routine.

%% full blown deets.
In order to implement an interrupt routine we have to enable the relevant interrupts and tell the interrupt controller where it can find the interrupt routine.
Interrupts are enabled for specific I/O units by loading the appropriate values into the Interrupt Enable Register.
We want to enable interrupts for SW0 and SW2.
Their masks are \texttt{0x1} and \texttt{0x4}, respectively.


We enable interrupts by loading the appropriate values into the Interrupt Enable Register (PER).
Specifically, the \"appropriate values\" are the masks for SW0 and SW2, which are \texttt{0x1} and \texttt{0x4}, respectively.
Since our buttons are connected to PIOB, we load the masks
More specifically, we load these masks into a register, and then 
\\*
We load PIO B's base address into r2
\texttt{
		/* */
		lddpc r2, piob_offset
		/* The masks of SW_0 and SW_2 are loaded into r5 */
		mov r5, SW_0 | SW_2
		/* */
		st.w r2[AVR32_PIO_IER], r5
}
% store the mask of SW0 and SW2 in r5

% starts a new line, but not a new paragraph
% PIO_IER is the interrupt enable register
\texttt{st.w r2[AVR32_PIO_IER], r5 }

This is done by storing the appropriate values in the PIO Enable Register (PER).
We enabled button interrupts for \texttt{SW0} and \texttt{SW2} by loading their masks into the PIOB, with the appropriate offset to specify
GPIO pins = PIO
general purpose input/output pins

% load the pointer to PIOB's base address into r2
lddpc r2, piob_offset
% load the pointer to the interrupt controller's base address into r7
lddpc r7, intc_base
% store the address of the interrupt routine in r8
mov r8, button_interrupt_routine
% store the mask of SW0 and SW2 in r5
mov r5, SW_0 | SW_2
% enable button interrupts for SW0 and SW2
% PIO_IER is the interrupt enable register
st.w r2[AVR32_PIO_IER], r5 
% stores the address of our button interrupt routine in the interrupt controller
st.w r7[AVR32_INTC_IPR14], r8

\texttt{st.w r2[AVR32_PIO_IER], r5}
Enables button interrupts for SW0 and SW2.
First, interrupts must be disabled


lddpc r7, intc_base
% intc_base is a pointer to the address of the interrupt controller

mov r8, button_interrupt_routine
% the address of the interrupt routine is loaded into r8

st.w r7[AVR32_INTC_IPR14], r8
% the address of the interrupt routine is stored in the interrupt controller's button h

The time is specified 
First the state of the buttons is read.
The debounce routine simply waits for a time decided by the DEBOUNCE constant in CONFIGURATION.

st.w rd, rs
% writes the value in rs to the address found in rd


et spesielt minneområde som skal inneholde adressen til en rutine som vil bli kjørt når kortet mottar en relevant interrupt. I dette tilfelle er relevante interrupts knappe-interrupts.

rcall read_buttons
rcall debounce

ld.w r0, r2[AVR32_PIO_ISR]

rd, rs

rete
