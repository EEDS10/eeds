All tests assume that you are in possession of at least one (1) functional STK1000 development board (with cables), a JTAGICE mkII (with USB cable) and a computer with software and hardware capable of interfacing with the JTAGICE.
\paragraph{Button Functionality Test}
This test aims to uncover if pushing Button 0 has the desired effect (the desired effect from pushing Button 0 is that the paddle moves one LED to the right.)
\\ Prerequisites:
\begin{itemize}
	\item One (1) finger
	\item Functional eyesight
\end{itemize}
Procedure:
\begin{enumerate}
	\item Upload the code to the board (e.g using \texttt{make upload}).
	\item Push the board's reset button.
	\item Note the paddle's position.
	\item Push Button 0.
	\item Note the paddle's position.
\end{enumerate}
If the paddle's position in the last step is not one to the right of its position in step \#3, Button 0 does not have the appropriate functionality.
The test can be refactored to test Button 2: simply push Button 2 instead in step \#4 and note that the test will have failed if the paddle's position in step \#5 is not one to the left of its position in step \#3.

\paragraph{Measurement of Power Consumption}
This test measures the board's power consumption by removing a jumper and connecting an amperemeter in series.
Power consumption is traditionally preferably measured while the board is turned on and 
\\
Prerequisites:
\begin{itemize}
	\item{Multimeter or amperemeter}
	\item{Hands, or other tool with similar prehensile ability}
\end{itemize}
Procedure:
\begin{enumerate}
	\item{Turn the STK1000 on and upload your code.}