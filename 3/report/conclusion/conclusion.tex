The Conclusion: can be very short in most 
undergraduate laboratories.
Simply state what you know now for sure, as a result of the lab: 

Example: The Debye-Sherrer method identified 
the sample material as nickel due to the measured 
crystal structure (fcc) and atomic radius 
(approximately 0.124 nm).

Notice that, after the material is identified in the 
example above, the writer provides a justification. 
We know it is nickel because of its structure and 
size. This makes a sound and sufficient 
conclusion. Generally, this is enough; however, 
the conclusion might also be a place to discuss 
weaknesses of experimental design, what future 
work needs to be done to extend your conclusions, 
or what the implications of your conclusion are.