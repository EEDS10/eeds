This section describes the solutions' various components.

\subsection{Linux}
	The assignment required that the Linux 2.6 kernel be compiled for the AVR32 STK1000 development board.
	Specifically, Linux 2.6.16.11-avr32-20060626 was used.

	Linux was compiled and flashed onto a \texttt{16GB} SD Card, by virtue of following the distributed guidelines and a great deal of blood, sweat and tears\footnote{there were so many tears.}.

\subsection{Linux Device Drivers for the STK1000}
	Something about LDDs?
	Init and Exit functions and cleanup and such

	\subsubsection{LED driver}
		the LED driver allows the software to turn the LEDs on and off?
	\subsubsection{button driver}
		the buttan driver reads the buttons' state.	

	\subsubsection{Sound driver}
		/dev/dsa!
	\subsubsection{Display driver}
		The STK1000's has a screen. It has a resolution of \texttt{320}\cross \texttt{240}.
		/dev/fb0
		Screen size: 320x240
		32 bits per pixel (8bit color depth)

\subsection{Game Engine -- WORKING TITLE}
	this is the game engine we developed. it has parts that are pretty good. like sound, it can play sound and display things on the screen.
	\subsubsection{Sound}
		how does the engine handle sound?
		we just pipe that to /dev/dsa right?
		pretty much I guess.
	\subsubsection{Graphics}
		them graphics are pretty good but how do we handle them????
\subsection{``Fontenizer'', the Font Engine}
	A font rendering engine (\texttt{font.h} and \texttt{font.c} found in \texttt{game/include/}) was developed in order to easily render some given text on the screen.
	It does this by iterating through a given string\footnote{Actually a C string.} and locating each character's corresponding glyph in a given font bitmap.
	The glyphs are strung together and passed to the screen buffer.

\subsection{The .sm file format}
	yo imma just rewrite http://www.stepmania.com/wiki/The_.SM_file_format