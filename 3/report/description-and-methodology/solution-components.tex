This section describes the solutions' various components.

\subsection{Linux}
The assignment required that the Linux 2.6 kernel be compiled for the AVR32 STK1000 development board.
Specifically, Linux 2.6.16.11-avr32-20060626 was used.

Linux was compiled and flashed onto a \texttt{2GB SanDisk UltraII} SD Card, by virtue of following the distributed guidelines and a great deal of blood, sweat and tears\footnote{there were so many tears.}.

\subsection{Linux Device Drivers for the STK1000}
In the game, STK1000's buttons are used to interface with the user.
The board's LEDs are also used to help with the interface's affordance, highlighting which buttons are available.
Both these actions require interfacing with the hardware of the STK1000, which is something that is unwise to do directly in an application-level program's code base, such as the code base of SLDDD.

This is where a device driver comes in handy. Its job is to provide an abstract interface to the hardware.
Loading device drivers is not straight forward, as they usually reside in kernel space.
Luckily, Linux makes it easy to dynamically load and remove drivers at will.

Many linux drivers, including the ones written for this assignment, allows other programs to interface with hardware through file-like devices.
The user may simply \texttt{fopen} a device, and then write and/or read data to it, using the driver as a middleware translator to translate the write and read calls to whatever is appropriate for the specific device.

\subsubsection{Loading drivers}
When kernel object (.ko) file has been compiled, it may be installed into the kernel with \texttt{insmod <driver>.ko}.
Device drivers require a file-like interface to the hardware, which is generated with the command \texttt{mknod \&dev/<device name> c <major number> <minor number>}.
The major number can be found in \texttt{/proc/devices}, and the minor number is \texttt{0} for the drivers written for this assignment.
A nifty one-liner for inserting a kernel module and making a device node for it as described above is shown in listing \ref{listing-insertmodule}.

LISTING linsting-insertmodule: \texttt{mknod /dev/<device name> c \$(grep <driver> /proc/devices | awk '\{print \$1\}') 0}

Later, to remove a driver (perhaps to preserve memory), it can be removed using \texttt{rmmod <driver>}, whilst also deleting the device file using \texttt{rm /dev/<device name>}.

\subsubsection{LED driver}
A custom driver was written for the STK1000 to facilitate the turning off and on of the LEDs.
On initialization the driver obtains a major number dynamically and requests a region of memory-mapped hardware before enabling the relevant I/O pins by writing \texttt{0xFF} to the PIO Enable Register (\texttt{PER}) and to the Output Enable Register (\texttt{OER}).

Upon exit, the driver turns the LEDs off by writing \texttt{0xFF} to the Clear Output Data Register (\texttt{CODR}) and \texttt{0x00} to the PIO Enable Register (\texttt{PER}) to disable the I/O pins.
It also releases the region it requested during initialization.

The driver interacts with the LEDs by first turning all the LEDs off (by writing \texttt{0xFF} to the \texttt{CODR}) and then enabling the desired LEDs by writing the given value to the Set Output Data Register (\texttt{SODR}).

For a more detailed explanation of the set-up of the LEDs on the STK1000, see \cite{tdt4258-1}.
For a reference C implementation of this, see \cite{tdt4258-2}.

Noteworthy is the fact that the LEDs no longer reside in PIOC, as in previous assignments \cite{tdt4258-1} and \cite{tdt4258-2}.
This is because PIOC is made unavailable by \texttt{SW6} on the STK1002, see \cite{lab-compendium}.
PIOB is therefore used for both buttons and LEDs.
The LEDs use some of the upper pins of PIOB -- see \cite{lab-compendium}.

\subsubsection{Button driver}
A custom driver was written in order to allow the software to read the buttons' state.

The button driver allows the video game to read the state of the buttons, letting them function as a source of input for the players.
Its initialization is similar to the LED driver's, except that it writes \texttt{0xFF} to the Pull-up Enable Register (\texttt{PUER}) rather than the \texttt{OER}.
Its exit function is similarly different.

The driver reads the button's state by returning the value found in the Pin-Data Status Register (\texttt{PDSR}).

The buttons use the bottom 8 PIOB pins.

For a more detailed explanation of the set-up of the buttons on the STK1000, see \cite{tdt4258-1}. Again, a C reference implementation might exists in \cite{4258-2} TODO: does it? Nope.

\subsubsection{Sound driver}
In order to produce sound, we use the provided ALSA\footnote{Advanced Linux Sound Architecture, see http://www.alsa-project.org/} driver.
This let us simply write raw sound data to \texttt{/dev/dsp} to play it, as described in \cite{lab-compendium}.

The installed linux defaults to muted audio settings, and the lab guidelines recommend using \texttt{alsamixer} to manually set the volume and un-mute it each time the board boots.
Because the authors prefer automating such work, a script is provided that programatically sets the correct audio settings using the more low level \texttt{alsactl}.

TODO: this is just a marker to remind us that we need to deliver the fix_sound.sh script.

\subsubsection{Display driver}
The STK1000 has an LCD screen from Samsung.
It has a resolution of \texttt{320}$\times$\texttt{240} and sports \texttt{32} bits per pixel (\texttt{24} bit color depth, the last \texttt{8} bits are unused)\cite{lab-compendium}.
However, only 24 of these are enabled by default\cite{avr32-disp}.

Data is displayed on the screen by writing a bitmap buffer directly to \texttt{/dev/fb0}.

% should we mention something about orientation?

    \subsection{EGE  - Eeds Game Engine}

    A simple state based game engine was developed specifically for this assignment to ease the development of the solution game.
    The game engine features a simple state system which allows separation of different parts of the program.
    Each state can be thought of almost as a separate program, with separate functions for initializing, deinitializing, updating and so on.
    In the solution game separate states are used for the \textit{splash screen}, the \textit{song selection menu}, the actual gameplay part, and the \textit{score screen}.

    The game engine also is responsible for timers, and makes sure that the game always runs at the same speed independent of CPU load.
    This is done by enforcing a separation of the logical update code and rendering code.
    The idea is that while logical updating at a fixed rate is critical to maintaining temporal consistency in a game, rendering is not.
    Should the logical updating falls behind the schedule, rendering is delayed until the logical updating can catch up with real time again.
    Finally, the game only performs a render if the logical code has signalled that the screen needs a refresh since the last time it was redrawn.
    This is done to save time that would be wasted on drawing the same image to the screen two update cycles in a row.

    The game engine also handles input and output through buttons and LEDs, providing convenient interfaces for the programmer to use.
    The buttons are regularly polled and their state is stored in a global array of boolean values.
    This allows the programmer to simply check the array to see if a button is pressed.
    Button presses can even be simulated programatically by writing to the global button array inbetween polling the buttons.

    The LEDs are exposed through a single function \texttt{set_leds}, which allows the programmer to turn the LEDs on and off.


    Finally, the game engine also supports simple audio playback of raw audio data files.

\subsubsection{Sound}
The sound module of the game engine is capable of playing single raw audio files, as well as controlling the soundcard to set parameters such as bitrate and frequency.
The game engine supports playback of one audio file at a time, which can be set by loading and unloading an audio file to a global pointer.
Mixing, compositing, and other advanced features have not been implented.

\subsubsection{Graphics}
The graphics rendering module is entirely based on bitmaps.
In order to generate the graphics to display to the user, various bitmaps are blitted in order to achieve the desired visual components and effects.


\subsection{``Fontenizer'': the Font Engine}
	A font rendering engine (\texttt{game/include/font.h} and \texttt{game/src/font.c}) was developed in order to easily render some given text on the screen.
	It does this by iterating through a given string\footnote{Actually a C string.} and locating each character's corresponding glyph in a given font bitmap.
	The glyphs are blitted one by one from the font bitmap to a destination bitmap one at a time.
    Variable horizontal character spacing is supported on a per-call basis, allowing for sophisticated kerning.
    Fontenizer supports the basic ASCII character set.
        
\subsection{The \texttt{.SM} file format}
The \texttt{.SM}\footnote{http://www.stepmania.com/wiki/The_.SM_file_format} ile format was created by the developers of StepMania, possibly to facilitate the creation and sharing of notes for songs.
An \texttt{.SM}-file contains the dance steps for a song, as well as metadata relating to the song itself (the name of the audio file, the name of the artist and such).

The game engine is capable of reading \texttt{.SM}-files and extracting the relevant data.
Implementing this file format was deemed a feasible alternative to creating notes by hand from scratch, as the popularity of StepMania means there are a wide range of publicly accessible StepMania songs available free of charge on the internet.