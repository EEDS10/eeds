\section{Energy Efficiency}
	the solution doesn't make use of interrupts.
	the buttons are polled each frame. some sort of interrupt-based scheme could surely be implemented.

	update() is called more often than render().

	cpu usage vs redrawing on the screen (is it better to redraw large areas or just the specific areas affected between frames?)

\section{Testing}
\subsection{Basic functionality test}

\subsection{Energy consumption}


\subsection{Game engine test suite}
	all those tests we made

	all those tests we made

	runnin' through my board

	runnin' through my board

	runnin' through my board

	(runnin' through my board)

	they were not enough
	\subsubsection{Evaluating the efficiency of functions}
		Select functions print their runtime/TTC (time to completion) to the console, allowing an evaluation to be made regarding their efficiency.
		This facilitates the evaluation of changes made to functions: if a change in the function results in an increase in its runtime, the change did not contribute to increased efficiency.

		This particular technique is a kind of profiling\footnote{\url{http://en.wikipedia.org/wiki/Profiling_(computer_programming)}} dubbed ``ghetto profiling'' by one of the authors. 

\subsection{Gameplay}
	we had people test the gameplay.
	those who did not enjoy themselves were eliminated.
	\subsubsection{Prototyping}
		rough test of gameplay
		two people attempted to control the STK1000's buttons at the same time, resting one finger on each button.
		After the conclusion of the test the participants reported that 
	\subsubsection{External reviews}
		we let some peeps play the finished game.
		they rated it somewhere between 0 and 100.
		the test involved having them play the game.


\section{Discussion}
	hey yo what up what is there to discuss?
	maybe we should have conducted more gameplay tests underway?