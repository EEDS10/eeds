\section{Energy Efficiency}
	the solution doesn't make use of interrupts.
	the buttons are polled each frame. some sort of interrupt-based scheme could surely be implemented.

	update() is called more often than render().

	cpu usage vs redrawing on the screen (is it better to redraw large areas or just the specific areas affected between frames?)

\section{Testing}
\subsection{Basic functionality tests}
	\subsubsection{LED \& button driver test}
		\begin{description}
			\item[Description] \hfill \\
				This test is designed to assertain whether the LED and button drivers function as intended.
			\item[Prerequisites] \hfill
				\begin{itemize}
					\item{a driver for the STK1000's LEDs.}
					\item{a driver for the STK1000's buttons.}
					\item{a working installation of Linux running on the STK1000.}
					\item{The LED driver must be able to parse the data produced by reading the buttons' state through the button driver.}
				\end{itemize}
			\item[Procedure] \hfill
				\begin{enumerate}
					\item{Load the LED \& button drivers using \texttt{insmod} and \texttt{mknod}.}
					\item{enter \texttt{cat /dev/buttons > /dev/leds} into the console.}
				\end{enumerate}
			\item[Expected results] \hfill \\
				When \texttt{SWn} is held down \texttt{LEDn} should light up.
				\\$n \in [0,7]$
		\end{description}	

\subsection{Energy consumption}
	what about it? Right we should probably test it.

\subsection{Game engine test suite}
	all those tests we made

	all those tests we made

	runnin' through my board

	runnin' through my board

	runnin' through my board

	(runnin' through my board)

	they were not enough
	\subsubsection{Evaluating the efficiency of functions}
		Select functions print their runtime/TTC (time to completion) to the console, allowing an evaluation to be made regarding their efficiency.
		This facilitates the evaluation of changes made to functions: if a change in the function results in an increase in its runtime, the change did not contribute to increased efficiency.

		This particular technique is a kind of profiling\footnote{\url{http://en.wikipedia.org/wiki/Profiling_(computer_programming)}} dubbed ``ghetto profiling'' by one of the authors. 

\subsection{Gameplay}
	we had people test the gameplay.
	those who did not enjoy themselves were eliminated.
	\subsubsection{Playtesting}
	\begin{description}
		\item[Description] \hfill \\
			Video games are a recreational activity.
			That means people play them in their spare time.
			Who would want to spend their spare time doing something that's not enjoyable?
			This test is designed to uncover what parts of the video game that are enjoyable and what parts are less enjoyable.
		\item[Prerequisites] \hfill
			\begin{itemize}
				\item{A working build of the solution.}
				\item{One or more live subjects.}
			\end{itemize}
		\item[Procedure] \hfill
			\begin{enumerate}
				\item{Introduce the subject to the solution.}
				\item{Allow the subject to attempt to interface with the solution.}
				\item{Wait.}
				\item{Collect feedback from subject.}
			\end{enumerate}
		\item[Expected results] \hfill \\
			Feedback from the participants.
	\end{description}


\section{Discussion}
	hey yo what up what is there to discuss?
	maybe we should have conducted more gameplay tests underway?

	we were originally going to implement two-player mode but when we simulated the gameplay (two people each trying to get their fingers on four buttons) we realized that it wasn't very comfortable.