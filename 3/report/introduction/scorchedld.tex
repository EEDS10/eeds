\begin{quote}
	In warfare, the policy of burning and destroying everything that might be of use to an invading army, especially the crops in the fields.\cite{lab-compendium}
\end{quote}
Scorched Land Defense (SLD) is a warfare tactic. The assignment requests that a computer game with the given title be written.

We can interpret the quoted definition as describing a situation in which one entity (A) attacks a defending entity (B) which, rather than defending itself directly, attempts to destroy its own resources.
A's objective thus becomes not simply the destruction of B, but also the prevention of B's scorched land defensive strategy while B's achieves victory by either successfully carrying out its defensive strategy or through eliminating the threat (i.e. the defeat of A).

Throughout the course of a ``game'' both entities perform actions with the intended goal of achieving victory.
The end of a game is a state in which neither side cannot perform any actions or any of the victory conditions are met.

In video games, the player(s) typically controls some entity and is able to select what actions the entity should perform through some interface, typically a keyboard, gamepad, controller or screen.

\subsection{Scorched Land Dance Dance Defense}
In Scorched Land Dance Dance Defense (SLDDD) the entities have four distinct available actions throughout the course of a game.
If performed within the specified time frame, the actions aid the entities in achieving their given victory conditions.
The players' objective thus becomes to perform as many successful actions as possible throughout the course of a game.

The time window in which a given action will be successful if performed is presented to the player as an object moving across the screen.
When the object reaches a specified point on the screen, the player must perform the action corresponding to its position.
The closer the object is to the specified point at the time when the player performs the correct action, the more successful the action is.
Successful actions result in the player receiving ``points''.
More successful actions are worth more points than less successful actions.

The player may choose which entity (or side) he or she is playing as.
If two or more players were to play at the same time, they could imagine themselves both being the same entity working towards the same goal or separate entities fighting against each other.
