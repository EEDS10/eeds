The STK1000 has 13 general purpose registers named \texttt{r0-r12}. The programmer is free to use these registers for whatever they want.
However, several conventions are commonplace to introduce a certain degree of structure.

\paragraph{Conventions}
\texttt{r0} is a scratch register, used for intermediate calculations and such.
\texttt{r1} holds the constant 0.
\texttt{r8-12} are used to hold parameters when calling a routine.
\texttt{r12} is used to hold the return value when returning from a routine.

On the basis of these conventions, a set of rules has been laid down governing what each register of the processor is used for.
This list of registers and their usage is presented in table \ref{register-table}.

\begin{table}
    \centering
    \begin{tabular}{|l|l|}
        \hline
        Register  & Usage \\
        \hline
        \hline
        \texttt{r0}  & Scratch register used to hold intermediate values. \\
        \hline
        \texttt{r1}  & Constant: \texttt{0} \\ 
        \hline
        \texttt{r2}  & Constant: base offset to \texttt{PIOB} \\ 
        \hline
        \texttt{r3}  & Constant: base offset to \texttt{PIOC} \\ 
        \hline
        \texttt{r4}  & Variable: holds the position of the paddle \\ 
        \hline
        \texttt{r5}  & Constant: \texttt{5} \\ 
        \hline
        \texttt{r6}  & Constant: \texttt{0xff} \\ 
        \hline
        \texttt{r7}  & Variable: holds the previous button state \\ 
        \hline
        \texttt{r8}  & Constant: pointer to button interrupt routine \\ 
        \hline
        \texttt{r9}  & (not used) \\ 
        \hline
        \texttt{r10} & (not used) \\ 
        \hline
        \texttt{r11} & (not used) \\ 
        \hline
        \texttt{r12} & Holds return value from routines \\
        \hline
    \end{tabular}
    \caption{Overview over register use}
    \label{register-table}
\end{table}
