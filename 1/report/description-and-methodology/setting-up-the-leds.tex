On the STK1000, the connection to the output LEDs must be set up before the LEDs can be used in a program.
First, the I/O pins that the LEDs are connected to must be enabled.
In this assignment, the LEDs were connected to the pins \texttt{GPIO16-23}, corresponding to \texttt{PIO C} lines \texttt{0-7}.
To enable the correct I/O pins, we must therefore set bits \texttt{0-7} of the \texttt{PIO C} PIO Enable Register (\texttt{PIOC PER}) to 1, as in listing 2.
Here, \texttt{r3} is the base address of \texttt{PIOC}, \texttt{AVR32\_PIO\_PER} is the PIO Enable Register offset, and \texttt{r6} contains the bit field indicating which pins to enable.

\code{99}{100}{Enable the I/O pins}

Second, the I/O pins must be set to act as output pins, as opposed to input pins.
This is done by setting the corresponding bits (\texttt{0-7}) of the \texttt{PIO C} Output Enable Register (\texttt{PIOC OER}) to 1, as in listing 3.
Here, \texttt{r3} is the base address of \texttt{PIOC}, \texttt{AVR32\_PIO\_OER} is the Output Enable Register offset, and \texttt{r6} contains the bit field indicating which pins to set as outputs.

\code{102}{103}{Set the pins to act as output pins}

Once this is done, LEDs can be turned on by writing the appropriate bits to \texttt{PIO C} Set Output Data Register (\texttt{PIOC SODR}), as in listing 4.
Here, \texttt{r3} is the address of \texttt{PIOC}, \texttt{AVR32\_PIO\_SODR} is the Set Output Data Register offset, and \texttt{r4} contains the bit field indicating which LEDs to switch on.

\code{176}{177}{Switch on LEDs}

Analogously, LEDs can be turned off by writing the appropriate bits to \texttt{PIO C} Clear Output Data Register (\texttt{PIOC CODR}), as in listing 5. 
Here, \texttt{r3} is the address of \texttt{PIOC}, \texttt{AVR32\_PIO\_SODR} is the Set Output Data Register offset, and \texttt{r4} contains the bit field indicating which LEDs to switch off.

\code{173}{174}{Switch off LEDs}
