
\documentclass{article}
%%%%%%%%%%%%%%%%%%%%%%%%%%%%%%%%%%%%%%%%%%%%%%%%%%%%%%%%%%%%%%%%%%%%%%%%%%%%%%%%%%%%%%%%%%%%%%%%%%%%%%%%%%%%%%%%%%%%%%%%%%%%%%%%%%%%%%%%%%%%%%%%%%%%%%%%%%%%%%%%%%%%%%%%%%%%%%%%%%%%%%%%%%%%%%%%%%%%%%%%%%%%%%%%%%%%%%%%%%%%%%%%%%%%%%%%%%%%%%%%%%%%%%%%%%%%
\usepackage{geometry}
\usepackage{fancyhdr}
\usepackage[pdftex]{graphicx}

%TCIDATA{OutputFilter=LATEX.DLL}
%TCIDATA{Version=5.50.0.2953}
%TCIDATA{<META NAME="SaveForMode" CONTENT="1">}
%TCIDATA{BibliographyScheme=Manual}
%TCIDATA{Created=Monday, January 30, 2012 17:20:46}
%TCIDATA{LastRevised=Monday, February 27, 2012 12:06:08}
%TCIDATA{<META NAME="GraphicsSave" CONTENT="32">}
%TCIDATA{<META NAME="DocumentShell" CONTENT="Standard LaTeX\Blank - Standard LaTeX Article">}
%TCIDATA{CSTFile=40 LaTeX article.cst}

\newtheorem{theorem}{Theorem}
\newtheorem{acknowledgement}[theorem]{Acknowledgement}
\newtheorem{algorithm}[theorem]{Algorithm}
\newtheorem{axiom}[theorem]{Axiom}
\newtheorem{case}[theorem]{Case}
\newtheorem{claim}[theorem]{Claim}
\newtheorem{conclusion}[theorem]{Conclusion}
\newtheorem{condition}[theorem]{Condition}
\newtheorem{conjecture}[theorem]{Conjecture}
\newtheorem{corollary}[theorem]{Corollary}
\newtheorem{criterion}[theorem]{Criterion}
\newtheorem{definition}[theorem]{Definition}
\newtheorem{example}[theorem]{Example}
\newtheorem{exercise}[theorem]{Exercise}
\newtheorem{lemma}[theorem]{Lemma}
\newtheorem{notation}[theorem]{Notation}
\newtheorem{problem}[theorem]{Problem}
\newtheorem{proposition}[theorem]{Proposition}
\newtheorem{remark}[theorem]{Remark}
\newtheorem{solution}[theorem]{Solution}
\newtheorem{summary}[theorem]{Summary}
\newenvironment{proof}[1][Proof]{\noindent\textbf{#1.} }{\ \rule{0.5em}{0.5em}}
% Sets page margins to 1", which is standard
\geometry{left=1in,right=1in,top=1in,bottom=1in} 
% allows the included extensions of graphic files
\DeclareGraphicsExtensions{.pdf,.png,.jpg}
% sets graphic path. Otherwise it just looks around
\graphicspath{{}}
% I do not remember what this does
\setlength{\headheight}{15.2pt}
% allows the xhead parameters
\pagestyle{fancy}
% replace with your name
\lhead{Group \#10}
% replace 
\rhead{Group 10 IS BEST GROUP}
% \input{tcilatex}
\begin{document}

% INCLUDEGRAPHICS EXPLANATION
% \includegraphics[scale=1]{name of file}
% sometimes you want to twice encase the filename in squiggly brackets. I do not know why but sometimes it is required.

% begin title page, use \\ for newline
\title{\#EEDS\\Report from Lab Assignment \#1\\TDT4258 Energy Efficient Computer Systems}
% now one can list the authors, \textbf{} makes bold text
\author{Emil Taylor Bye}
\author{Sigve Sebastian Farstad}
\author{Odd M. Trondrud}
% make title page
\maketitle

% create new page
\newpage

% uncomment this if you want to include a ToC
% \tableofcontents
% disables chapter, section and subsection numbering
% set to 1 or higher to enable section numbering
% it goes something like 0:Part, 1:Section, 2:Section, 3:Subsection, 4:Subsubsection, ...
\setcounter{secnumdepth}{-1}

% use \newpage to get newpage
% \bigskip to get some space between paragraphs
% \texttt{TEXTHERE} to get typewriter-like (basically just monospaced) text

\part{Abstract}

This report is the solution to assignment 1 of TDT4258 at NTNU, spring 2013.
In the assigment, we program an Atmel STK1000 development board to display a moveable LED 'paddle' using AVR32 assembly and the GNU tool chain.

\part{Introduction}

This is the introduction.

\part{Description and Methodology}

This is the description and methodology.

Configuration of the STK1000
============================


Programming environment
=======================

JTAGICE
-------

GNU Debugger
------------

Make, other tools etc
---------------------

Development of the program
==========================

Setting up the LED diodes
-------------------------

Setting up the buttons
----------------------

Interrupt Routine
-----------------

Refactoring and Modularization
------------------------------


\part{Results and Tests}

This is the results and tests section.

Energy Efficiency
=================

Testing
=======

\part{Evaluation of Assignment}

Here comes the evaluation of the assignment.

\part{Conclusion}

This is the place for the conclusion.

\part{Acknowledgement}

Here, we acknowledge those who deserve to be acknowledged.

\part{References}

And finally, references.

\end{document}
